%______________________________________________________
%
%   LaTeX-mall för nybörjare
%
%   Konstruerad av Marcus Bergner, bergner@cs.umu.se
%
%   Vid funderingar titta längst ned i denna fil,
%   eller skicka ett mail
%______________________________________________________
%

% lite inställningar
\documentclass[11pt, titlepage, oneside, a4paper]{article}
\usepackage[T1]{fontenc}
\usepackage[utf8]{inputenc}
\usepackage[english]{babel}
\usepackage{amssymb, graphicx, fancyhdr, amsmath}
\usepackage{hyperref}
\addtolength{\textheight}{20mm}
\addtolength{\voffset}{-5mm}
\renewcommand{\sectionmark}[1]{\markleft{#1}}

% \Section ger mindre spillutrymme, använd dem om du vill
\newcommand{\Section}[1]{\section{#1}\vspace{-8pt}}
\newcommand{\Subsection}[1]{\vspace{-4pt}\subsection{#1}\vspace{-8pt}}
\newcommand{\Subsubsection}[1]{\vspace{-4pt}\subsubsection{#1}\vspace{-8pt}}
	
% appendices, \appitem och \appsubitem är för bilagor
\newcounter{appendixpage}

\newenvironment{appendices}{
	\setcounter{appendixpage}{\arabic{page}}
	\stepcounter{appendixpage}
}

\newcommand{\appitem}[2]{
	\stepcounter{section}
	\addtocontents{toc}{\protect\contentsline{section}{\numberline{\Alph{section}}#1}{\arabic{appendixpage}}}
	\addtocounter{appendixpage}{#2}
}

\newcommand{\appsubitem}[2]{
	\stepcounter{subsection}
	\addtocontents{toc}{\protect\contentsline{subsection}{\numberline{\Alph{section}.\arabic{subsection}}#1}{\arabic{appendixpage}}}
	\addtocounter{appendixpage}{#2}
}

% Ändra de rader som behöver ändras
\def\inst{Computing Science}
\def\typeofdoc{Assignment report}
\def\course{Emergent Systems VT-14,\\7,5 hp}
\def\pretitle{Assignment 1}
\def\title{Termites and terminators}
\def\name{Emil Edskär and Martin Lärka}
\def\username{id10eer}
\def\usernameB{id10mla}
\def\email{\username{}@cs.umu.se}
\def\emailB{\usernameB{}@cs.umu.se}
\def\path{edu/KURS/labNR}
\def\graders{Jonny Pettersson}


% Om du vill referera till katalogen där dina filer ligger kan du 
% använda \fullpath som kommer att vara "~username/edu..." o.s.v.
\def\fullpath{\raisebox{1pt}{$\scriptstyle \sim$}\username/\path}


% Här brjar själva dokumentet
\begin{document}

	% Skapar framsidan (om den inte duger: gör helt enkelt en egen)
	\begin{titlepage}
		\thispagestyle{empty}
		\begin{large}
			\begin{tabular}{@{}p{\textwidth}@{}}
				\textbf{UMEÅ UNIVERSITET \hfill \today} \\
				\textbf{Institutionen för \inst} \\
				\textbf{\typeofdoc} \\
			\end{tabular}
		\end{large}
		\vspace{10mm}
		\begin{center}
			\LARGE{\pretitle} \\
			\huge{\textbf{\course}}\\
			\vspace{10mm}
			\LARGE{\title} \\
			\vspace{15mm}
			\begin{large}
				\begin{tabular}{ll}
					\textbf{Name} & \name \\
					\textbf{Mail} & \texttt{\email} and \texttt{\emailB}\\
				\end{tabular}
			\end{large}
			\vfill
			\large{\textbf{Tutorer}}\\
			\mbox{\large{\graders}}
		\end{center}
	\end{titlepage}


	% Fixar sidfot
	\lfoot{\footnotesize{\name}}
	\rfoot{\footnotesize{\today}}
	\lhead{\sc\footnotesize\title}
	\rhead{\nouppercase{\sc\footnotesize\leftmark}}
	\pagestyle{fancy}
	\renewcommand{\headrulewidth}{0.2pt}
	\renewcommand{\footrulewidth}{0.2pt}

	% Skapar innehållsförteckning.
	% Tänk på att köra latex 2ggr för att uppdatera allt
	\pagenumbering{roman}
	\tableofcontents
	
	% och lägger in en sidbrytning
	\newpage

	\pagenumbering{arabic}

	% I Sverige har vi normalt inget indrag vid nytt stycke
	\setlength{\parindent}{0pt}
	% men däremot lite mellanrum
	\setlength{\parskip}{10pt}

	% Lägger in rubrik (finns \section, men då får man mycket spillutrymme)
	\Section{Bakgrund}
	I denna laboration undersöks emergenta egenskaper för ett system med termiter i en simulerad datorvärld. För att åstadkomma detta har programmet NetLogo använts. NetLogo är ett verktyg som är byggt för att simulera komplexa system som utvecklas med tiden.

	Systemet som undersökts simulerar en värld delvis fylld av termiter och träbitar. Termiterna följer reglerna:
		\begin{enumerate}
			\item Vandrar slumpmässigt runt (utan att bära någon träbit).
			\item Om träbit hittas, tar upp den.
			\item Vandrar slumpmässigt runt (bär med en träbit).
			\item Om träbit hittas, lägger ner träbiten som den redan bär på.
			\item Börja om från steg 1.
		\end{enumerate}

	Genom att varje termit i systemet följer dessa regler kommer systemet som helhet att nå den emergenta egenskapen, alla träbitar samlas i en och samma hög.

	I systemet finns även termiter som motverkar systemet. Alltså termiter som följer dessa regler:
		\begin{enumerate}
			\item Vandrar slumpmässigt runt (utan att bära någon träbit).
			\item Om träbit hittas, tar upp den.
			\item Springer iväg fyrtio steg i slumpmässigt riktning (bär med en träbit).
			\item Vandra omkring och letar yta utan träbitar.
			\item Om inga träbitar finns runtom den tomma ytan, lägg ner träbiten.
			\item Vandra iväg tjugo steg i slumpmässig riktning.
			\item Börja om från steg 1.
		\end{enumerate}

	Frågan som tas upp i denna laboration är hur stor andel av de termiter som motverkar systemet som måste finnas i systemt för att den emergenta egenskapen, alla träbitar i samma hög, inte ska uppstå.

	Utöver detta undersöks även om det blir någon skillnad om en till sorts samlande termit introduseras i systemet. Denna termit samlar en annan sorts träbitar och försöker skapa en separat hög av dessa.

	\Section{Metoder} % Martin
	
	As evident from \autoref{fig1} $r$ is a very good approximation of tan within the interval $J_{4} = \left(-\frac{\pi}{32},\frac{\pi}{32}\right)$
	
	%\includegraphics[scale=0.5]{p2_01.png}\label{fig1}
	
%         $f(x) = \exp(x), f(x) = \alpha$
% 		\begin{align*}
%                   x = \log(\alpha) \Rightarrow f(x) = f(\log(\alpha)) = \exp(\log(\alpha)) = \alpha
%                 \end{align*}
% 		
%         Obviously $\log(\alpha)$ is a solution to $f(x) = \alpha$. It also the unique solution to the equation since
%         \begin{align*}
%           &g(x) = f(x) - \alpha = e^{x}-\alpha\\
%           &g'(x) = e^{x} > 0, \forall x \in \mathbb{R}
%         \end{align*}
% 
%         $g(x)$ has no stationary points which means there can only exist one solution to the equation $g(x) = 0$ (according to Rolle's theorem).

\Section{Resultat} % Emil
	
	

\Section{Diskussion} % Båda


\newpage
\Section{Källkod}



\end{document}


% Lite information om hur man arbetar med LaTeX
%-----------------------------------------------
%
% LaTeX-koden kan skrivas med en godtycklig editor.
% För att "kompilera" dokumentet anvnds kommandot latex:
%    bergner@peppar:~/edu/sysprog/lab1> latex rapportmall.tex
% Resultatet blir ett antal filer, bl.a. en som heter rapportmall.dvi.
% Denna fil kan anvndas för att titta hur dokumentet egentligen ser
% ut med hjlp av programmet xdvi:
%    bergner@peppar:~/edu/sysprog/lab1> xdvi rapportmall.dvi &
% Du får då upp ett fönster som visar ditt dokument. Detta fönster
% kommer automatiskt att uppdateras då du ändrar och kompilerar om din
% LaTeX-kod. 
% Nr du anser att din rapport är färdig att skrivas ut anvnder man
% lämpligtvis kommandona dvips och lpr:
%    bergner@peppar:~/edu/sysprog/lab1> dvips -P ma436ps rapportmall.dvi
% Om man vill ha kvar PostScript-filen som dvips genererar kan man göra:
%    bergner@peppar:~/edu/sysprog/lab1> dvips -o rapport.ps rapportmall.dvi
%    bergner@peppar:~/edu/sysprog/lab1> lpr -P ma436ps rapport.ps
% OBS!!! För att innehållsförteckningen och eventuella referenser till
% tabeller och figurer garanterat ska stämma mäste man köra latex 2ggr
% på sitt dokument efter att man har ändrat ngot.
%
%
% Lite information om saker man kan tänkas behöva i sitt arbete med LaTeX
%-------------------------------------------------------------------------
%
% FORMATTERA TEXT
%
% För att formattera text på lite olika sätt kan man använda följande LaTeX-
% kommandon:
%    \textbf{denna text kommer att vara i fetstil}
%    \emph{denna text är viktig (kursiv stil)}
%    \texttt{i denna text blir alla tecken lika breda, som med en skrivmaskin}
%    \textsf{denna text visas med ett typsnitt utan serifer}
%
%
% MATEMATISKA FORMLER
%
% För att typsätta matematiska formler kan man använda:
%    $f(x) = x^2 - 3$, vilket lägger in formeln i texten, eller
%    \begin{displaymath}
%        g(x) = \frac{\sin x}{x}
%    \end{displaymath}, vilket låter formeln visas centrerat på en egen rad
% Om du vill att en formel ska numreras byter du ut displaymath mot equation.
% Det finns massor med matematiska symboler, vilket gör att man behver
% någon liten manual att titta i om man ska konstruera avancerade formler.
% Se slutet på filen för lite råd om var du kan hitta sådana.
%
%
% INFOGA FIGURER
%
% För att infoga en figur kan man göra på följande sätt:
%    \begin{figure}[htb]
%        \includegraphics[scale=0.5, angle=90]{exec_flow.eps}
%        \caption{Detta r bildtexten}
%        \label{EXECFLOW}
%    \end{figure}
% Om man vill referera till denna bild i texten skulle man då skriva enligt:
%    ...i figur \ref{EXECFLOW} kan man se att...
% Några små förklaringar till figurer:
%    [htb] = talar om hur latex ska försöka placera bilden (Here, Top, Bottom)
%            Om du anvnder [!h], innebär det Here!!!
%    scale = kan skala om bilden, om den är skalbar
%    angle = kan rotera bilden
%    exec_flow.eps = filnamnet på bilden. Notera att formatet .EPS används
% För att skapa figurer används lämpligtvis programmet xfig:
%    bergner@peppar:~/edu/sysprog/lab1> xfig &
% Rita (och spara ofta) tills du är klar. Välj sedan "Export" och exportera
% din figur till EPS-format.
% Om man vill kan man använda endast \includegraphics, men det är inte ofta
% man gär det.
%
%
% INFOGA TABELLER
%
% Om man vill skapa en tabell gör man på följande sätt:
%    \begin{table}[htb]
%        \begin{tabular}{|rlp{10cm}|}
%            \hline
%            13 & $17.26$ & En kommentar som kan strcka sig ver flera rader \\
%            \hline
%        \end{tabular}
%        \caption{Tabelltexten...}
%        \label{TBL:MINTABELL}
%    \end{table}
% Om man vill kan man endast använda raderna 2-6, dvs få en tabell utan text
% och nummer. Om man gör på detta vis kommer tabellen alltid att läggas på
% det ställe den skrivs i koden, dvs ungefär samma sak som [!h] -> Here!!!
% Några frklaringar:
%    l, r, c = vänsterjustera, högerjustera eller centrera kolumn
%    p{bredd} = skapa en vänsterjusterad kolumn med en viss bredd
%               kan innehålla flera rader text
%    | = en vertikal linje i tabellen
%    \hline = en horisontell linje i tabellen
%    & = kolumnseparator
%    \\ = radseparator
% Tänk på att tabeller oftast ser bättre ut med ganska få linjer.
%
%
% INFOGA KÄLLKOD ELLER UTDATA FRÅN TESTKRNINGAR
%
% Om man vill infoga källkod eller något annat liknande, t.ex. utdata från
% en testkörning är det bra om LaTeX återger utdatan korrekt, dvs en radbrytning
% betyder en radbrytning och 8 mellanslag på rad betyder 8 mellanslag på rad.
% För att stadkomma detta anvnds:
%    \begin{verbatim}
%        allt som skrivs här återges exakt, med skrivmaskinstypsnitt
%    \end{verbatim}
% Oftast finns det dock bttre verktyg för att skriva ut källkod. Exempel på
% sådana är a2ps, enscript och atp.
%
%
% ÄNDRA STORLEK PÅ TEXT
%
% Om du vill ändra storleken på ett stycke, t.ex. på din nyss infogade
% testkörning omger du stycket med \begin{STORLEK} \end{STORLEK}, där
% STORLEK är någon av:
%    tiny, scriptsize, footnotesize, small, normalsize, large, Large,
%    LARGE, huge, Huge
% Tänk på att inte mixtra för mycket med storlekar bara.
%
%
% SKAPA LISTOR AV OLIKA SLAG
%
% Det är ganska vanligt att man vill rada upp saker på något sätt. För att
% skapa punktlistor används:
%    \begin{itemize}
%        \item Detta är första punkten
%        \item Detta är andra punkten
%    \end{itemize}
% Om man istället vill ha en numrerad lista kan man anvnda enumerate istället
% för itemize. Listor kan anvndas i flera nivåer
%
%
% MER INFORMATION OM LaTeX
%
% Lite blandad information om LaTeX, länkar och annat hittar du på
% http://www.cs.umu.se/~bergner/latex.htm
% En del information om rapportskrivning hittar du på
% http://www.cs.umu.se/~bergner/rapport/
% Det finns massor med information om LaTeX på Internet. Ett litet urval:
% http://www.giss.nasa.gov/latex/
%     är en mycket välfylld sida om LaTeX
% http://wwwinfo.cern.ch/asdoc/WWW/essential/essential.html
%     är en manual som genererats utifrån ett LaTeX-dokument mha latex2html
% http://tex.loria.fr/english/
%     är ett fylligt arkiv av länkar till LaTeX-dokument på Internet
%
% Min personliga favorit är dock manualen "The Not So Short Introduction to
% LaTeX2e", som finns i DVI-format på ~bergner/LaTeX/lshort2e.dvi
% Där står i princip allt man behver veta. Det är bara att använda xdvi och
% titta efter det du söker, vilket oftast finns där.
% Om du, precis som jag, vill kunna leka med många kommandon i LaTeX finns en
% "LaTeX Command Summary" p ~bergner/LaTeX/latexcmds.ps
