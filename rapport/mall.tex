%______________________________________________________
%
%   LaTeX-mall för nybörjare
%
%   Konstruerad av Marcus Bergner, bergner@cs.umu.se
%
%   Vid funderingar titta längst ned i denna fil,
%   eller skicka ett mail
%______________________________________________________
%

% lite inställningar
\documentclass[11pt, titlepage, oneside, a4paper]{article}
\usepackage[T1]{fontenc}
\usepackage[utf8]{inputenc}
\usepackage[swedish]{babel}
\usepackage{amssymb, graphicx, fancyhdr, amsmath, hyphenat}
\usepackage{hyperref}

\addtolength{\textheight}{20mm}
\addtolength{\voffset}{-5mm}
\renewcommand{\sectionmark}[1]{\markleft{#1}}

% \Section ger mindre spillutrymme, använd dem om du vill
\newcommand{\Section}[1]{\section{#1}\vspace{-8pt}}
\newcommand{\Subsection}[1]{\vspace{-4pt}\subsection{#1}\vspace{-8pt}} 
\newcommand{\Subsubsection}[1]{\vspace{-4pt}\subsubsection{#1}\vspace{-8pt}}
	
% appendices, \appitem och \appsubitem är för bilagor
\newcounter{appendixpage}

\newenvironment{appendices}{
	\setcounter{appendixpage}{\arabic{page}}
	\stepcounter{appendixpage}
}

\newcommand{\appitem}[2]{
	\stepcounter{section}
	\addtocontents{toc}{\protect\contentsline{section}{\numberline{\Alph{section}}#1}{\arabic{appendixpage}}}
	\addtocounter{appendixpage}{#2}
}

\newcommand{\appsubitem}[2]{
	\stepcounter{subsection}
	\addtocontents{toc}{\protect\contentsline{subsection}{\numberline{\Alph{section}.\arabic{subsection}}#1}{\arabic{appendixpage}}}
	\addtocounter{appendixpage}{#2}
}

% Ändra de rader som behöver ändras
\def\inst{Computing Science}
\def\typeofdoc{Assignment report}
\def\course{Emergent Systems VT-14,\\7,5 hp}
\def\pretitle{Projekt}
\def\title{Ultimata strategin i Paradise Hotel-finalen}
\def\name{Emil Edskär, Martin Lärka, Johan Holmgren och Peter Bjuhr}
\def\footname{Edskär, Lärka, Holmgren och Bjuhr}
\def\username{id10eer}
\def\usernameB{id10mla}
\def\usernameC{id10jhn}
\def\usernameD{id10pbl}
\def\email{\username{}@cs.umu.se}
\def\emailB{\usernameB{}@cs.umu.se}
\def\emailC{\usernameC{}@cs.umu.se}
\def\emailD{\usernameD{}@cs.umu.se}
\def\path{edu/KURS/labNR}
\def\graders{Jonny Pettersson}


% Om du vill referera till katalogen där dina filer ligger kan du 
% använda \fullpath som kommer att vara "~username/edu..." o.s.v.
\def\fullpath{\raisebox{1pt}{$\scriptstyle \sim$}\username/\path}


% Här brjar själva dokumentet
\begin{document}

	% Skapar framsidan (om den inte duger: gör helt enkelt en egen)
	\begin{titlepage}
		\thispagestyle{empty}
		\begin{large}
			\begin{tabular}{@{}p{\textwidth}@{}}
				\textbf{UMEÅ UNIVERSITET \hfill \today} \\
				\textbf{Institutionen för \inst} \\
				\textbf{\typeofdoc} \\
			\end{tabular}
		\end{large}
		\vspace{10mm}
		\begin{center}
			\LARGE{\pretitle} \\
			\huge{\textbf{\course}}\\
			\vspace{10mm}
			\LARGE{\title} \\
			\vspace{15mm}
			\begin{large}
				\begin{tabular}{ll}
					\textbf{Name} & \name \\
					\textbf{Mail} & \texttt{\email} \texttt{\emailB} \\
					& \texttt{\emailC} \texttt{\emailD} \\
				\end{tabular}
			\end{large}
			\vfill
			\large{\textbf{Tutorer}}\\
			\mbox{\large{\graders}}
		\end{center}
	\end{titlepage}


	% Fixar sidfot
	\lfoot{\footnotesize{\footname}}
	\rfoot{\footnotesize{\today}}
	\lhead{\sc\footnotesize\title}
	\rhead{\nouppercase{\sc\footnotesize\leftmark}}
	\pagestyle{fancy}
	\renewcommand{\headrulewidth}{0.2pt}
	\renewcommand{\footrulewidth}{0.2pt}

	% Skapar innehållsförteckning.
	% Tänk på att köra latex 2ggr för att uppdatera allt
	\pagenumbering{roman}
	\tableofcontents
	
	% och lägger in en sidbrytning
	\newpage

	\pagenumbering{arabic}

	% I Sverige har vi normalt inget indrag vid nytt stycke
	\setlength{\parindent}{0pt}
	% men däremot lite mellanrum
	\setlength{\parskip}{10pt}

	%\begin{abstract}
	%Here goes the actual text of your abstract. 
	%åÅäÄöÖ
	%\end{abstract}

	% Lägger in rubrik (finns \section, men då får man mycket spillutrymme)

	\section{Inledning}
	Ett välkänt tv-program i Sverige, kallat “Paradise Hotel”, utser sin vinnare genom ett psykologiskt spel som spelas mellan två finaldeltagare som tilldelas varsin boll. Spelet består utav totalt 10 tidssteg, där spelarna kan välja att släppa eller att behålla bollen vid varje tidssteg. Varje tidssteg representeras av en prissumma, som ökar linjärt vid varje steg. Om båda deltagarna behåller bollen genom alla tidssteg delar de jämnt på det sista tidsstegets prissumma. Om en spelare väljer att släppa bollen på ett visst tidssteg erhåller denne hela prissumman för aktuellt tidssteg, medans motståndaren går lottlös. Med andra ord kan en spelare vinna hälften av pengarna genom att släppa bollen efter halva tiden eller genom att hålla i bollen spelet ut, men samtidigt riskera att bli blåst.

Detta projekt avser att undersöka en iterativ variant av detta spel, där spelarna möter varandra många gånger och där prissumman ersätts utav poäng. Detta är intressant eftersom de spelare som väljer att vara godtrogna vid varje spel, är de spelare som troligtvis vinner minst antal gånger samtidigt som de kammar hem fler poäng per vinst. De spelare som släpper bollen vid första tidssteget vid varje spel kommer att vinna varje gång, men antalet poäng kommer att vara mindre. Målet är att hitta den strategi som presterar bäst över tid. 15 stycken strategier (spelare) har implementerats och tävlat mot varandra. Rapporten beskriver resultatet av följande frågor:

\begin{itemize}
\item \nohyphens{Vilken strategi ger maximal utdelning om spelet körs x antal omgångar?}\\*
	\emph{Hypotes: Strategin “Tit-for-tat”, som släpper bollen ett tidssteg tidigare än motståndarens senaste “drag”.}
\item Vilken strategi vinner flest omgångar?\\*
	\emph{Hypotes: Strategin som släpper vid tidssteg 1 vid varje spel.}
\item \nohyphens{Är det samma strategi som vinner flest omgångar som också får maximal utdelning i poäng?}\\*
	\emph{Hypotes: Nej, spelare som släpper vid tidiga tidssteg kommer att vinna flest gånger, men de vinner för lite poäng varje gång.}
\end{itemize}

	\newpage

	\section{Bakgrund}
	\subsection{Historia, game theory}
Spelteori skapades och strukturerades av John von Neumann and Oskar Morgenstern som en model av hur genomtänkta, rationella och potentiellt vilseledande spelare med motsatta intressen kommer agera mot varandra i ett visst spel~\cite{Flake:2001}. Detta ger möjlighet att sätta upp fördelaktiga strategier som maximerar vinsten eller minimerar förlusten till olika spel.

Till en början var spelteori begränsad till nollsummespel\footnote{Ett nollsummespel definieras som ett spel där summan av vinsten och förlusten alltid förblir noll för varje tänkbar strategi spelarna kan använda sig av.} mellan två spelare. Men under 50-talet utvidgades spelteorin av matematikern John Nash till att även ge möjlighet att analysera spel med flera spelare eller som inte är nollsummespel. 

För att undersöka tänkbara anpassningar av spelet som studeras i denna rapport har efterforskning gjorts för att hitta liknande spel inom området av spelteori. Det finns framförallt två spel som innehar stora likheter med spelet i denna rapport. Det första heter the ultimatum game och innefattar två spelare som interagerar med varandra för att dela på en summa pengar. Det andra spelet är en variant av the ultimatum game som kallas the dictator game. Även detta spel spelas mellan två spelare. Den ena agerar diktator och bestämmer på egen hand hur pengarna ska delas mellan de två spelarna.

Ett sätt att exemplifiera the ultimatum game är att tänka sig att två personer står och väntar på bussen. En tredje person kommer fram och erbjuder person nummer ett 1000 kronor, men bara på ett villkor: Han måste erbjuda person nummer två en del av pengarna. Han får välja själv hur mycket eller hur litet han vill erbjuda person nummer två, men om person nummer två inte går med på uppdelningen så får ingen av dem några pengar.

The ultimatum game togs fram av spelteoretikern Ariel Rubinstein i artikeln Perfect Equilibrium in a Bargaining Model~\cite{Rubinstein:1982}. Hans hypotes var att alla skulle välja att dela med sig av minsta möjliga summa till sin medspelare för att själva tjäna maximalt. Detta visade sig dock inte stämma när tre andra forskare, Werner Güth, Rolf Schmittberger och Bernd Schwarze~\cite{Werner:1982}, utförde ett experiment av the ultimatum game. Experimentet visade att person nummer ett i genomsnitt erbjöd 40-50\% av pengarna till sin medspelare samt att person nummer två i hälften av fallen avfärdade ett erbjudande som låg under 30\% av totalsumman.

The dictator game togs fram av Daniel Kahneman~\cite{Kahneman:1986}. I Detta spel agerar en av spelarna diktator och får välja helt fritt hur en summa pengar ska delas upp mellan honom själv och medspelaren. Spelaren som tar emot pengarna kan inget göra för att påverka hur mycket diktatorn delar med sig av. 

Kahneman hade en teori om att de spelare som intog rollen som diktator i detta spel skulle välja det rationella valet att ge bort så lite som möjligt av belöningen till sin medspelare, och behålla så mycket som möjligt själv. För att testa denna teori utförde Kahneman en studie av spelet där båda  spelarna som deltog var anonyma. Detta möjliggjorde att deltagarna kunde göra sina val helt utan rädsla att bli dömda av sin medspelare. Studien visade till Kahneman’s förvåning att de allra flesta deltagarna var generösare än förväntat och det var inte helt ovanligt att diktatorn delade med sig av hälften av belöningen.

Både the ultimatum game och the dictator game kan till viss del jämföras med det spel vi analyserar i denna rapport. Alla tre spelen innefattar två spelare vars mål är att vinna pengar (eller någon form av belöning). Alla tre spelen har ett rationellt val att försöka maximera sin egen belöning.

The ultimatum game har dessutom liknelsen att det innehåller ett avvägnings moment där högre vinst medför en större risk att bli utan belöning helt (den som gapar efter mycket mister ofta hela stycket). Även samarbete mellan spelarna är något som the ultimatum game innehåller. Samarbete medför att båda spelarna får ta del av vinsten, båda vinner något, men att ingen av dem får maximal utdelning.

De största skillnaderna är framförallt att the ultimatum game spelas med öppna kort. Båda spelarna ser vad den andra gör. Ytterligare skillnader är hur makten är fördelad mellan speldeltagarna i de olika spelen. I the ultimatum game är det en spelare som har makten att erbjuda en viss summa pengar till sin medspelaren. I vårt spel är det spelledaren som erbjuder båda spelarna samma summa exakt samtidigt.

\subsection{Verklighetsanknytning}
Att koppla vårt spel till något verkligt scenario (utöver finalen i paradise hotel) är svårt. Det går emellertid att hitta kopplingar till verkligheten för vissa av de spel som vi jämfört med. The ultimatum game kan liknas vid olika former av förhandlingar. Här kommer några exempel:

1. Ett företag erbjuder de anställda ett avtal och facket kan välja att antingen acceptera avtalet eller att förkasta det. Här måste företaget hitta ett jämnläge där båda parterna blir nöjda, annars kan företaget riskerar att en strejk bryter ut och varken företaget eller arbetarna tjänar något. 

2. Två länder under krigsförhandlingar kan ses som ett exempel på the ultimatum game. Om länderna inte kan komma överens om ett fredsförslag så kommer stridigheterna att fortsätta vilket ingen av parterna vinner på.~\cite{UltimatumGame}



\subsection{Klassificering}
Ett sätt att börja forma en spelteori är att klassificera vilken typ av spel det handlar om. ”Paradise Hotel”-spelet klassificeras som ett icke-nollsummespel som spelas mellan två motståndare. Det är inte ett nollsummespel då vinsten som går till den vinnande spelaren inte tas från dess motspelare utan från ”banken”. Det är ett symmetriskt spel då vinsten inte är beroende på vilken av spelarna som vinner. Eftersom att båda spelarna agerar samtidigt i spelet klassificeras spelet också att vara simultant. Det betyder att det inte är någon turordning i spelet, utan att båda spelarna har möjlighet att agera när de vill under spelets gång. Att beakta är även att båda spelarna har möjlighet att vinna om de samarbetar med varandra, dock så innebär detta att de inte maximerar sin egen potentiella vinst. Samt att spelarna inte har möjlighet att ta del av all information i spelet. De har bara möjlighet att veta vid vilka tidssteg som deras motståndare har släppt på om motståndaren vann, och vid vilka tidssteg som de själva har släppt på.

	\newpage

	\section{Metod}
	\subsection{Inledande analys}
För att genomföra denna undersökning och jämföra olika strategier mot varandra, inleddes implementationsdelen med en analys av spelet för att vara säker på att alla regler efterföljdes. Detta skedde genom att studera en video från en final, då ingen textdokumentation gick att återfinna.

Efter att ha klartgjort alla regler för gruppmedlemmarna bestod nästa del av en gemensam gruppdiskussion för att bestämma hur testet skulle utformas. Då tidigare laborationer utförts i NetLogo, låg detta program nära i tankarna för implementationen av detta test. Fördelarna med kraftfulla funktioner, en bra testmiljö och den enkla visualisationen av resultat gjorde att valet blev relativt enkelt. Parallellt med detta utformades de strategier som skulle testas mot varandra. 

Då programmet att utföra testet i var bestämt, påbörjades själva arbetet. Gruppen delades upp i undergrupper med två medlemmar vardera. En grupp fokuserade på att skapa miljön där strategierna ställs mot varandra, samt att ta fram och visualisera resultat. Den andra sattes på att implementera strategier. De båda grupperna samarbetade dock genom diskussioner under arbetets gång för att försöka skapa miljön och implementera strategierna på bästa sätt.

Testet består av 15 st strategier. Dessa strategier spelar mot alla andra strategier (inklusive sin egen). När de spelar sparar agenterna sin egen historik mot de andra strategierna, de andra strategiernas historik mot den egna strategin, hur många poäng strategin har samt hur många matcher den har vunnit. Strategier som tar hjälp av motståndarens historik kan dock bara titta på dessa resultat ifall motståndaren vann eller matchen slutade lika.

En match går till genom att två strategier möter varandra. De båda meddelar vilken nivå de har tänkt släppa glaskulan på, där en nivå i detta fall motsvaras av en siffra mellan ett och tio. Släpper båda på nivån tio ges båda fem poäng och matchen räknas som vunnen. Andra likaresultat räknas som lika men ger inga poäng. I annat fall vinner den strategi som valde att släppa tidigare än den andra, och motsvarande antal poäng utdelas till den vinnande strategin. Den vinnande strategin registrerar då också en vinst.

\subsection{Implementation}
För att uppnå detta skapas i detta fall 15 st agenter i NetLogo som var och en motsvarar en strategi. Strategierna är dock implementerade utanför agenterna vilket gör att agenterna inte möter varandra. Istället har varje agent en separat uppsättning matcher och resultat vilka inte är kopplade till de andra agenterna. Agenterna innehåller alla en lista med sina egna tidigare val, en lista med motståndarens tidigare val, en poängräknare och en räknare som håller koll på antalet vunna matcher. Både listan med egna val och listan med motståndarens val innehåller i sin tur en lista för varje strategi som agenten ska möta.

Då agenterna har skapats sätter själva testet igång, då alla agenter gås igenom i en slumpmässig ordning och spelar i turordning en match mot alla strategier. Ett tidssteg motsvarar alltså en match för alla strategier mot alla strategier (inklusive sig själva). 

Agenternas identifikation (“who”) står för vilken strategi de motsvarar. Matcherna spelas genom funktionen challenge som tar in två ID:n. Det första motsvarar den agent vars strategi nu spelar mot alla strategier och det andra vilken strategi som agentens strategi möter för tillfället. Utifrån dessa beräknas de båda strategiernas drag genom funktionen calc-move. Denna funktion anropar i sin tur de aktuella strategierna tillsammans med de båda strategiernas historik för den aktuella agenten, för att avgöra på vilken nivå som strategierna skulle ha släppt. De båda strategiernas drag jämförs sedan.

Om de båda strategierna skulle ha släppt på nivå tio delas fem poäng ut till den aktuella agenten och matchen räknas som vunnen vilket gör att ett adderas till räknaren för vunna matcher. Släpper de båda strategierna på samma nivå, som inte är den sista, räknas matchen som lika. Vinner agenten läggs den nivå agentens strategi skulle ha släppt på till i poängräknaren och ett adderas till vinsträknaren. Vinner motståndaren händer ingenting. Oavsett resultat läggs den nivå som strategierna valt till i historiken.

De strategier som tittar på motståndarens historik kan bara titta på de resultat där det antingen blivit lika eller där motståndaren vunnit. Detta görs genom funktionen result-list som tar in den egna historiken, motståndarens historik och en siffra som visar hur långt man vill kolla. Siffran som motsvarar hur långt man vill kolla är till för de strategier som bara tittar på den senaste matchen för att slippa gå igenom hela historiken.

Result-list skapar två nya listor från listorna med historik, där längden bestäms av den siffra man skickade in. Funktionen går sedan igenom de nya resultatlistorna och jämför varje värdepar för sig. Värden läggs sedan till i en ny lista enligt nedan. Om den egna strategins valda värde är lägre än motståndarens betyder det att den egna strategin vann rundan, och 1 läggs till på motsvarande plats i en ny lista. Är värdena lika läggs 0 till på motsvarande plats. Hade den andra strategin ett lägre värde innebär det en förlust och att -1 läggs till i den nya listan. Ett exempel på hur detta fungerar visas i Tabell \ref{table:result-list}.

\begin{table}[htb]
	\begin{center}
		\begin{tabular}{| l | l | l | l |}
			\hline
			Mitt val & 6 & 4 & 8 \\ \hline
			Motståndarens val & 7 & 4 & 2 \\ \hline
			Resultatlista & 1 & 0 & -1 \\ \hline
		\end{tabular}
	\end{center}
	\caption{Exempel för result-list}
	\label{table:result-list}
\end{table}

Strategierna som tittar på motståndarens historik använder sig sedan av denna resultatlista för att titta på vilka positioner det finns värden i motståndarens historiklista som det är tillåtet att titta på.

För att avgöra vilken strategi som är bäst är det alltså möjligt att titta både på vilken strategi som får mest poäng och vilken strategi som har vunnit flest matcher.

\subsection{Strategier}
En kort förklaring för varje strategi och dess beteende.

\subsubsection{Tit for tat}
Denna strategi är en anpassad variant på tit for tat från Prisoner’s dilemma och fungerar på följande sätt. Första matchen väntar denna strategi till nivå tio. I efterföljande matcher spelar strategin samma som den förra omgången vid lika eller vinst. Förlorar strategin spelar den ett mindre än motståndarens förra val. Är motståndarens förra val mindre eller lika med två spelar strategin ett.

\subsubsection{Static}
Denna strategi spelar alltid sex.

\subsubsection{Random}
När denna strategi anropas spelas ett slumpmässigt värde mellan ett och tio.

\subsubsection{It’s something}
Spelar alltid fyra.

\subsubsection{Scumbag Steve}
Strategin spelar den återkommande sekvensen 5-4-3-2-1-5-4...

\subsubsection{Scumbag Stacy}
Är en variant på Scumbag Steve men spelar istället sekvensen 5-3-1-5-3...

\subsubsection{Good Guy Greg}
Spelar alltid tio.

\subsubsection{Neil deGrasse Tyson}
Spelar fem så länge strategin vinner eller spelar lika. Direkt motståndaren har historik som strategin får titta på, spelar denna strategi ett avrundat medelvärde av motståndarens val.

\subsubsection{Robocop}
Spelar alltid fem.

\subsubsection{Close enough}
Spelar fem så länge motståndaren inte har tre resultat som den kan titta på, spelar sedan ett avrundat medelvärde av de tre senaste valen som motståndaren gjort.

\subsubsection{Even numbers}
Slumpar fram ett jämnt tal varje omgång.

\subsubsection{Mode}
Denna strategi spelar ett slumpmässigt värde tills motståndaren har ett värde som den kan titta på. Sedan väljer den ett under motståndarens typvärde. Har motståndaren flera typvärden väljer strategin det minsta och spelar ett under det.

\subsubsection{Median}
Spelar ett slumpmässigt värde tills motståndaren har ett värde som den kan titta på. När detta hänt spelar denna strategi ett under motståndarens medianvärde (som avrundas ifall ett jämnt antal matcher spelas och medianvärdet blir ett medelvärde av två värden).

\subsubsection{Grudger}
Spelar tio så länge motståndaren också spelar tio, från och med att motståndaren spelar något annat spelar denna strategi ett.

\subsubsection{Adjust}
Denna strategi spelar fem första omgången. Sedan ökar den ett vid vinst, spelar samma vid lika och minskar ett vid förlust.

	\newpage

	\section{Resultat}
	\begin{figure}[htb]
	\begin{center}
	\includegraphics[scale=0.75, angle=0]{bilder/points.png}
	\caption{Detta är bildtexten}
	\label{points}
	\end{center}
\end{figure}

\begin{figure}[htb]
	\begin{center}
	\includegraphics[scale=0.75, angle=0]{bilder/wins.png}
	\caption{Detta är bildtexten}
	\label{wins}
	\end{center}
\end{figure}

\begin{figure}[htb]
	\begin{center}
	\includegraphics[scale=0.75, angle=0]{bilder/adjust-guy.png}
	\caption{Detta är bildtexten}
	\label{adjust-guy}
	\end{center}
\end{figure}

\begin{figure}[htb]
	\begin{center}
	\includegraphics[scale=0.75, angle=0]{bilder/good-guy-greg.png}
	\caption{Detta är bildtexten}
	\label{good-guy-greg}
	\end{center}
\end{figure}

\begin{figure}[htb]
	\begin{center}
	\includegraphics[scale=0.75, angle=0]{bilder/grudger.png}
	\caption{Detta är bildtexten}
	\label{grudger}
	\end{center}
\end{figure}

\begin{figure}[htb]
	\begin{center}
	\includegraphics[scale=0.75, angle=0]{bilder/median-guy.png}
	\caption{Detta är bildtexten}
	\label{median-guy}
	\end{center}
\end{figure}

\begin{figure}[htb]
	\begin{center}
	\includegraphics[scale=0.75, angle=0]{bilder/tit-for-tat.png}
	\caption{Detta är bildtexten}
	\label{tit-for-tat}
	\end{center}
\end{figure}

	\newpage

	\section{Diskussion}
	\subsection{Resultat}

\subsection{Strategier}
Strategierna har utvecklats utav oss inom gruppen och har under projektets gång ifrågasatts en del. Vissa strategier har upplevts som meningslösa. I efterhand hade vi kunnat låta personer utanför projektgruppen skapa strategier, exempelvis som en tävling för att verkligen finna intressanta strategier, och framför allt fler strategier som tar hänsyn till resultathistoriken.\\

\noindent Ett intressant test som vi gjorde var att köra 10 strategier som släpper på varsitt tidssteg mot varandra flera gånger, dvs en strategi som släpper på 1, en som släpper på 2 och så vidare ända upp till 10. I detta test vann strategin som släppte på tidssteg 5 över tid. Strategi 4 och 6 hamnade på delad andra plats. Däremot så vann strategi nummer 1 över strategi nummer 5 i det “egentliga” testet. Därmet konstaterar vi att strategierna presterar väldigt olika beroende på vilka strategier som de möter.

\subsection{Projektet}
Gruppen har fungerat bra, och vi har följt tidsplanen och besvarat våra frågor. Vi kan emellertid tycka att vi skulle hittat relevant forskning i ett tidigare skede och utgå mer utifrån denna. Det var svårt att hitta artiklar som undersökt liknande frågeställningar. Vi tycker också att det var svårt att hitta vettiga problemformuleringar och frågeställningar inom just detta. Projektet startades egentligen bara med att vi diskuterade spelet och tyckte att det lät intressant att undersöka och då hamnade frågeställningarna i andrahand. Detta har inneburit att vi utvecklat vår medvetenhet gällande vad som krävs för att skriva vetenskapliga artiklar, där stor vikt skall läggas på tidigare forskning, samt formulerade av frågeställningar och hypoteser. I övrigt har projektet i sig har varit roligt att genomföra och har även gett oss ytterligare förståelse i framför allt game theory. 

\subsection{Framtida studier?}
Skall detta vara med också kanske??
	
	%\includegraphics[scale=0.5]{p2_01.png}\label{fig1}
	
%         $f(x) = \exp(x), f(x) = \alpha$
% 		\begin{align*}
%                   x = \log(\alpha) \Rightarrow f(x) = f(\log(\alpha)) = \exp(\log(\alpha)) = \alpha
%                 \end{align*}
% 		
%         Obviously $\log(\alpha)$ is a solution to $f(x) = \alpha$. It also the unique solution to the equation since
%         \begin{align*}
%           &g(x) = f(x) - \alpha = e^{x}-\alpha\\
%           &g'(x) = e^{x} > 0, \forall x \in \mathbb{R}
%         \end{align*}
% 
%         $g(x)$ has no stationary points which means there can only exist one solution to the equation $g(x) = 0$ (according to Rolle's theorem).
\pagebreak
%%%%% BEGIN OF 3rd ADJUSTMENTS SECTION %%%%%
% For the outline and annotated bibliography (deliverable 2)
% you must use the following two commands:
%
\nocite{*}  % Includes ALL entries from the .bib-file, even if they are
            % not '\cite{}'-ed in the text above
\bibliographystyle{plain-annote}
%
%For the full and final paper (deliverables 3 and 4a/b)
% use ONLY the following commands (i.e. put the commands above into
% comments and uncomment the command below):
%
% \bibliographystyle{splncs}
%
%%%%% END OF ADJUSTMENTS SECTION %%%%%
\bibliography{mall}


\end{document}


% Lite information om hur man arbetar med LaTeX
%-----------------------------------------------
%
% LaTeX-koden kan skrivas med en godtycklig editor.
% För att "kompilera" dokumentet anvnds kommandot latex:
%    bergner@peppar:~/edu/sysprog/lab1> latex rapportmall.tex
% Resultatet blir ett antal filer, bl.a. en som heter rapportmall.dvi.
% Denna fil kan anvndas för att titta hur dokumentet egentligen ser
% ut med hjlp av programmet xdvi:
%    bergner@peppar:~/edu/sysprog/lab1> xdvi rapportmall.dvi &
% Du får då upp ett fönster som visar ditt dokument. Detta fönster
% kommer automatiskt att uppdateras då du ändrar och kompilerar om din
% LaTeX-kod. 
% Nr du anser att din rapport är färdig att skrivas ut anvnder man
% lämpligtvis kommandona dvips och lpr:
%    bergner@peppar:~/edu/sysprog/lab1> dvips -P ma436ps rapportmall.dvi
% Om man vill ha kvar PostScript-filen som dvips genererar kan man göra:
%    bergner@peppar:~/edu/sysprog/lab1> dvips -o rapport.ps rapportmall.dvi
%    bergner@peppar:~/edu/sysprog/lab1> lpr -P ma436ps rapport.ps
% OBS!!! För att innehållsförteckningen och eventuella referenser till
% tabeller och figurer garanterat ska stämma mäste man köra latex 2ggr
% på sitt dokument efter att man har ändrat ngot.
%
%
% Lite information om saker man kan tänkas behöva i sitt arbete med LaTeX
%-------------------------------------------------------------------------
%
% FORMATTERA TEXT
%
% För att formattera text på lite olika sätt kan man använda följande LaTeX-
% kommandon:
%    \textbf{denna text kommer att vara i fetstil}
%    \emph{denna text är viktig (kursiv stil)}
%    \texttt{i denna text blir alla tecken lika breda, som med en skrivmaskin}
%    \textsf{denna text visas med ett typsnitt utan serifer}
%
%
% MATEMATISKA FORMLER
%
% För att typsätta matematiska formler kan man använda:
%    $f(x) = x^2 - 3$, vilket lägger in formeln i texten, eller
%    \begin{displaymath}
%        g(x) = \frac{\sin x}{x}
%    \end{displaymath}, vilket låter formeln visas centrerat på en egen rad
% Om du vill att en formel ska numreras byter du ut displaymath mot equation.
% Det finns massor med matematiska symboler, vilket gör att man behver
% någon liten manual att titta i om man ska konstruera avancerade formler.
% Se slutet på filen för lite råd om var du kan hitta sådana.
%
%
% INFOGA FIGURER
%
% För att infoga en figur kan man göra på följande sätt:
%    \begin{figure}[htb]
%        \includegraphics[scale=0.5, angle=90]{exec_flow.eps}
%        \caption{Detta r bildtexten}
%        \label{EXECFLOW}
%    \end{figure}
% Om man vill referera till denna bild i texten skulle man då skriva enligt:
%    ...i figur \ref{EXECFLOW} kan man se att...
% Några små förklaringar till figurer:
%    [htb] = talar om hur latex ska försöka placera bilden (Here, Top, Bottom)
%            Om du anvnder [!h], innebär det Here!!!
%    scale = kan skala om bilden, om den är skalbar
%    angle = kan rotera bilden
%    exec_flow.eps = filnamnet på bilden. Notera att formatet .EPS används
% För att skapa figurer används lämpligtvis programmet xfig:
%    bergner@peppar:~/edu/sysprog/lab1> xfig &
% Rita (och spara ofta) tills du är klar. Välj sedan "Export" och exportera
% din figur till EPS-format.
% Om man vill kan man använda endast \includegraphics, men det är inte ofta
% man gär det.
%
%
% INFOGA TABELLER
%
% Om man vill skapa en tabell gör man på följande sätt:
%    \begin{table}[htb]
%        \begin{tabular}{|rlp{10cm}|}
%            \hline
%            13 & $17.26$ & En kommentar som kan strcka sig ver flera rader \\
%            \hline
%        \end{tabular}
%        \caption{Tabelltexten...}
%        \label{TBL:MINTABELL}
%    \end{table}
% Om man vill kan man endast använda raderna 2-6, dvs få en tabell utan text
% och nummer. Om man gör på detta vis kommer tabellen alltid att läggas på
% det ställe den skrivs i koden, dvs ungefär samma sak som [!h] -> Here!!!
% Några frklaringar:
%    l, r, c = vänsterjustera, högerjustera eller centrera kolumn
%    p{bredd} = skapa en vänsterjusterad kolumn med en viss bredd
%               kan innehålla flera rader text
%    | = en vertikal linje i tabellen
%    \hline = en horisontell linje i tabellen
%    & = kolumnseparator
%    \\ = radseparator
% Tänk på att tabeller oftast ser bättre ut med ganska få linjer.
%
%
% INFOGA KÄLLKOD ELLER UTDATA FRÅN TESTKRNINGAR
%
% Om man vill infoga källkod eller något annat liknande, t.ex. utdata från
% en testkörning är det bra om LaTeX återger utdatan korrekt, dvs en radbrytning
% betyder en radbrytning och 8 mellanslag på rad betyder 8 mellanslag på rad.
% För att stadkomma detta anvnds:
%    \begin{verbatim}
%        allt som skrivs här återges exakt, med skrivmaskinstypsnitt
%    \end{verbatim}
% Oftast finns det dock bttre verktyg för att skriva ut källkod. Exempel på
% sådana är a2ps, enscript och atp.
%
%
% ÄNDRA STORLEK PÅ TEXT
%
% Om du vill ändra storleken på ett stycke, t.ex. på din nyss infogade
% testkörning omger du stycket med \begin{STORLEK} \end{STORLEK}, där
% STORLEK är någon av:
%    tiny, scriptsize, footnotesize, small, normalsize, large, Large,
%    LARGE, huge, Huge
% Tänk på att inte mixtra för mycket med storlekar bara.
%
%
% SKAPA LISTOR AV OLIKA SLAG
%
% Det är ganska vanligt att man vill rada upp saker på något sätt. För att
% skapa punktlistor används:
%    \begin{itemize}
%        \item Detta är första punkten
%        \item Detta är andra punkten
%    \end{itemize}
% Om man istället vill ha en numrerad lista kan man anvnda enumerate istället
% för itemize. Listor kan anvndas i flera nivåer
%
%
% MER INFORMATION OM LaTeX
%
% Lite blandad information om LaTeX, länkar och annat hittar du på
% http://www.cs.umu.se/~bergner/latex.htm
% En del information om rapportskrivning hittar du på
% http://www.cs.umu.se/~bergner/rapport/
% Det finns massor med information om LaTeX på Internet. Ett litet urval:
% http://www.giss.nasa.gov/latex/
%     är en mycket välfylld sida om LaTeX
% http://wwwinfo.cern.ch/asdoc/WWW/essential/essential.html
%     är en manual som genererats utifrån ett LaTeX-dokument mha latex2html
% http://tex.loria.fr/english/
%     är ett fylligt arkiv av länkar till LaTeX-dokument på Internet
%
% Min personliga favorit är dock manualen "The Not So Short Introduction to
% LaTeX2e", som finns i DVI-format på ~bergner/LaTeX/lshort2e.dvi
% Där står i princip allt man behver veta. Det är bara att använda xdvi och
% titta efter det du söker, vilket oftast finns där.
% Om du, precis som jag, vill kunna leka med många kommandon i LaTeX finns en
% "LaTeX Command Summary" p ~bergner/LaTeX/latexcmds.ps
