\subsection{Resultat}

\subsection{Strategier}
Strategierna har utvecklats utav oss inom gruppen och har under projektets gång ifrågasatts en del. Vissa strategier har upplevts som meningslösa. I efterhand hade vi kunnat låta personer utanför projektgruppen skapa strategier, exempelvis som en tävling för att verkligen finna intressanta strategier, och framför allt fler strategier som tar hänsyn till resultathistoriken.\\

\noindent Ett intressant test som vi gjorde var att köra 10 strategier som släpper på varsitt tidssteg mot varandra flera gånger, dvs en strategi som släpper på 1, en som släpper på 2 och så vidare ända upp till 10. I detta test vann strategin som släppte på tidssteg 5 över tid. Strategi 4 och 6 hamnade på delad andra plats. Däremot så vann strategi nummer 1 över strategi nummer 5 i det “egentliga” testet. Därmet konstaterar vi att strategierna presterar väldigt olika beroende på vilka strategier som de möter.

\subsection{Projektet}
Gruppen har fungerat bra, och vi har följt tidsplanen och besvarat våra frågor. Vi kan emellertid tycka att vi skulle hittat relevant forskning i ett tidigare skede och utgå mer utifrån denna. Det var svårt att hitta artiklar som undersökt liknande frågeställningar. Vi tycker också att det var svårt att hitta vettiga problemformuleringar och frågeställningar inom just detta. Projektet startades egentligen bara med att vi diskuterade spelet och tyckte att det lät intressant att undersöka och då hamnade frågeställningarna i andrahand. Detta har inneburit att vi utvecklat vår medvetenhet gällande vad som krävs för att skriva vetenskapliga artiklar, där stor vikt skall läggas på tidigare forskning, samt formulerade av frågeställningar och hypoteser. I övrigt har projektet i sig har varit roligt att genomföra och har även gett oss ytterligare förståelse i framför allt game theory. 

\subsection{Framtida studier?}
Skall detta vara med också kanske??