\subsection{Historia, game theory}
Spelteori skapades och strukturerades av John von Neumann and Oskar Morgenstern som en model av hur genomtänkta, rationella och potentiellt vilseledande spelare med motsatta intressen kommer agera mot varandra i ett visst spel. Detta ger möjlighet att sätta upp fördelaktiga strategier som maximerar vinsten eller minimerar förlusten till olika spel.\\

\noindent Till en början var spelteori begränsad till nollsummespel\footnote{Ett nollsummespel definieras som ett spel där summan av vinsten och förlusten alltid förblir noll för varje tänkbar strategi spelarna kan använda sig av.} mellan två spelare. Men under 50-talet utvidgades spelteorin av matematikern John Nash till att även ge möjlighet att analysera spel med flera spelare eller som inte är nollsummespel. 

\noindent För att undersöka tänkbara anpassningar av spelet som studeras i denna rapport har efterforskning gjorts för att hitta liknande spel inom området av spelteori. Det finns framförallt två spel som innehar stora likheter med spelet i denna rapport. Det första heter the ultimatum game och innefattar två spelare som interagerar med varandra för att dela på en summa pengar. Det andra spelet är en variant av the ultimatum game som kallas the dictator game. Även detta spel spelas mellan två spelare. Den ena agerar diktator och bestämmer på egen hand hur pengarna ska delas mellan de två spelarna.\\

\noindent Ett sätt att exemplifiera the ultimatum game är att tänka sig att två personer står och väntar på bussen. En tredje person kommer fram och erbjuder person nummer ett 1000 kronor, men bara på ett villkor: Han måste erbjuda person nummer två en del av pengarna. Han får välja själv hur mycket eller hur litet han vill erbjuda person nummer två, men om person nummer två inte går med på uppdelningen så får ingen av dem några pengar.\\

\noindent The ultimatum game togs fram av spelteoretikern Ariel Rubinstein. Han hypotes var att alla skulle välja att dela med sig av minsta möjliga summa till sin medspelare för att själva tjäna maximalt. Detta visade sig dock inte stämma när tre andra forskare, Werner Güth, Rolf Schmittberger och Bernd Schwarze, utförde ett experiment av the ultimatum game. Experimentet visade att person nummer ett i genomsnitt erbjöd 40-50\% av pengarna till sin medspelare samt att person nummer två i hälften av fallen avfärdade ett erbjudande som låg under 30\% av totalsumman.\\

\noindent Güth, Werner, Rolf Schmittberger, and Bernd Schwarze (1982) “An Experimental Analysis of Ultimatum Bargaining,” Journal of Economic Behavior and Organization, 3:4 (December), 367-388.\\

\noindent The dictator game togs fram av Daniel Kahneman (Fairness and the Assumptions of Economics). I Detta agerar en av spelarna diktator och får välja helt fritt hur en summa pengar ska delas upp mellan honom själv och medspelaren. Spelaren som tar emot pengarna kan inget göra för att påverka hur mycket diktatorn delar med sig av. \\

\noindent Kahneman hade en teori om att de spelare som intog rollen som diktator i detta spel skulle välja det rationella valet att ge bort så lite som möjligt av belöningen till sin medspelare, och behålla så mycket som möjligt själv. För att testa denna teori utförde Kahneman en studie av spelet där båda  spelarna som deltog var anonyma. Detta möjliggjorde att deltagarna kunde göra sina val helt utan rädsla att bli dömda av sin medspelare. Studien visade till Kahneman’s förvåning att de allra flesta deltagarna var generösare än förväntat och det var inte helt ovanligt att diktatorn delade med sig av hälften av belöningen.\\

\noindent Både the ultimatum game och the dictator game kan till viss del jämföras med det spel vi analyserar i denna rapport. Alla tre spelen innefattar två spelare vars mål är att vinna pengar (eller någon form av belöning). Alla tre spelen har ett rationellt val att försöka maximera sin egen belöning.\\

\noindent The ultimatum game har dessutom liknelsen att det innehåller ett avvägnings moment där högre vinst medför en större risk att bli utan belöning helt (den som gapar efter mycket mister ofta hela stycket). Även samarbete mellan spelarna är något som the ultimatum game innehåller. Samarbete medför att båda spelarna får ta del av vinsten, båda vinner något, men att ingen av dem får maximal utdelning.\\

\noindent De största skillnaderna är framförallt att the ultimatum game spelas med öppna kort. Båda spelarna ser vad den andra gör. Ytterligare skillnader är hur makten är fördelad mellan speldeltagarna i de olika spelen. I the ultimatum game är det en spelare som har makten att erbjuda en viss summa pengar till sin medspelaren. I vårt spel är det spelledaren som erbjuder båda spelarna samma summa exakt samtidigt.\\

\subsection{Real life}
\noindent Att koppla vårt spel till något verkligt scenario (utöver finalen i paradise hotel) är svårt. Det går emellertid att hitta kopplingar till verkligheten för vissa av de spel som vi jämfört med. The ultimatum game kan liknas vid olika former av förhandlingar. Här kommer några exempel: \\

1. Ett företag erbjuder de anställda ett avtal och facket kan välja att antingen acceptera avtalet eller att förkasta det. Här måste företaget hitta ett jämnläge där båda parterna blir nöjda, annars kan företaget riskerar att en strejk bryter ut och varken företaget eller arbetarna tjänar något. \\

2. Två länder under krigsförhandlingar kan ses som ett exempel på the ultimatum game. Om länderna inte kan komma överens om ett fredsförslag så kommer stridigheterna att fortsätta vilket ingen av parterna vinner på.\\

%http://www.econport.org/econport/request?page=man_tfr_experiments_ultimatumgame \\


\subsection{Klassificering}
\noindent Ett sätt att börja forma en spelteori är att klassificera vilken typ av spel det handlar om. ”Paradise Hotel”-spelet klassificeras som ett icke-nollsummespel som spelas mellan två motståndare. Det är inte ett nollsummespel då vinsten som går till den vinnande spelaren inte tas från dess motspelare utan från ”banken”. Det är ett symmetriskt spel då vinsten inte är beroende på vilken av spelarna som vinner. Eftersom att båda spelarna agerar samtidigt i spelet klassificeras spelet också att vara simultant. Det betyder att det inte är någon turordning i spelet, utan att båda spelarna har möjlighet att agera när de vill under spelets gång. Att beakta är även att båda spelarna har möjlighet att vinna om de samarbetar med varandra, dock så innebär detta att de inte maximerar sin egen potentiella vinst. Samt att spelarna inte har möjlighet att ta del av all information i spelet. De har bara möjlighet att veta vid vilka tidssteg som deras motståndare har släppt på om motståndaren vann, och vid vilka tidssteg som de själva har släppt på.